\documentclass{article}
\usepackage{amsmath}
\usepackage{amsfonts}
\usepackage{geometry}
\usepackage[russian]{babel}
\usepackage{fancyhdr}
\usepackage{titlesec}
\usepackage{indentfirst}
\geometry{a4paper, margin=1in}

% Отключаем нумерацию на титульном листе
\pagenumbering{gobble}

\begin{document}

    \title{\textbf{Формальная постановка задачи оптимизации расписания с использованием алгоритма имитации отжига}}
    \author{Кяжин Никита Олегович}
    \date{\today}
    \maketitle

    % Включаем нумерацию страниц после титульного листа
    \newpage
    \pagenumbering{arabic}
    \pagestyle{fancy}
    \fancyhf{}
    \rhead{\thepage}

    \section*{Формальная постановка задачи}
        
        \subsection*{Дано:}
            \begin{itemize}
                \item Пусть $J = \{j_1, j_2, \dots, j_N\}$ -- множество заданий, где $N$ -- количество заданий.
                \item Пусть $\tau  = \{t_1, t_2, \dots, t_N\}$ -- для каждого задания $j_i$ задано время выполнения $t_i > 0$.
                \item Пусть $P = \{p_1, p_2, \dots, p_M\}$ -- множество процессоров, на которых выполняются задания, где $M$ -- количество процессоров.
            \end{itemize}
        
        \subsection*{Расписание:}
            Расписанием является булева матрица $S^{N \times M}$, в которой $s_{ij} \in \{0, 1\}$, где $i \in {1, \dots, N}$, а $j \in {1, \dots, M}$. Значение $s_{ij} = 1$ означает,
            что задание $i$ выполняется на процессоре $j$, а $s_{ij} = 0$ -- что задание $i$ не выполняется на процессоре $j$.
            
            Обозначим $G_j$ - множество индексов задач, которые выполняются $j$ процессором. Тогда 
            $T_j = \Sigma_{i \in G_j}t_i$ -- время выполнения всех задач, запланированных на $j$ процессор.

        \subsection*{Требуется:}
            Построить расписание $S^{N \times M}$, при котором будет минизирован критерий, при этом все задания $J$ будут выполнены на множестве процессоров $P$ без прерываний,
            с учетом ограниченных ресурсов, и не будет пересечений в использовании процессоров.

        \subsection*{Минимизируемый критерий:}
            В зависимости от остатка от деления на 2 контрольной суммы CRC32 от фамилии и инициалов выбирается один из следующих критериев:
            \begin{itemize}
                \item Критерий $K_1$ (разбалансированность расписания)
                \item Критерий $K_2$ (суммарное время ожидания)
            \end{itemize}
            
            $CRC32_{KiazhinNO} = 3618506679$, следовательно выбираем 1 критерий для реализации.

            \subsubsection*{Критерий разбалансированности расписания:}

            \begin{equation}
                K_1 = T_{max} - T_{min}
            \end{equation}
            где:
            \begin{equation}
                T_{max}= \max_{j \in {1, \dots, M}}T_j
            \end{equation}
            \begin{equation}
                T_{min} = \min_{j \in {1, \dots, M}}T_j
            \end{equation}
    \section*{Ограничения}

        \begin{itemize}
            \item Каждый процессор $p_j \in P$ в любой момент времени может выполнять не более одного задания.
            \item Во время выполнения задания процессором, не возникает прерываний.
            \item Процессор может мгновенно (без прерывания) переключаться между заданиями.
            \item Каждое задание $j_i \in J$ должно быть выполнено только один раз и только на одном процессоре.
            \item Время выполнения $t_i \in \tau$ фиксировано.
        \end{itemize}

\end{document}