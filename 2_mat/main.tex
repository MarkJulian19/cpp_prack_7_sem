\documentclass{article}
\usepackage{amsmath}
\usepackage{amsfonts}
\usepackage{geometry}
\usepackage[russian]{babel}
\usepackage{fancyhdr}
\usepackage{titlesec}
\usepackage{indentfirst}
\geometry{a4paper, margin=1in}

% Отключаем нумерацию на титульном листе
\pagenumbering{gobble}

\begin{document}

    \title{\textbf{Формальная постановка задачи оптимизации расписания с использованием алгоритма имитации отжига}}
    \author{Кяжин Никита Олегович}
    \date{\today}
    \maketitle

    % Включаем нумерацию страниц после титульного листа
    \newpage
    \pagenumbering{arabic}
    \pagestyle{fancy}
    \fancyhf{}
    \rhead{\thepage}

    \section*{Формальная постановка задачи}
        
        \subsection*{Дано:}
            \begin{itemize}
                \item Пусть $J = \{j_1, j_2, \dots, j_N\}$ -- множество заданий, где $N$ -- количество заданий.
                \item Пусть $\tau  = \{t_1, t_2, \dots, t_N\}$ -- для каждого задания $j_i$ задано время выполнения $t_i > 0$.
                \item Пусть $P = \{p_1, p_2, \dots, p_M\}$ -- множество процессоров, на которых выполняются задания, где $M$ -- количество процессоров.
            \end{itemize}
        
        \subsection*{Расписание:}
            Расписанием является булева матрица $S^{N \times M}$, в которой $s_{i,j} \in \{0, 1\}$, где $i$ находится в диапазоне от $1$ до $N$, а $j$ -- в диапазоне от $1$ до $M$. Значение $s_{i,j} = 1$ означает,
            что задание $i$ выполняется на процессоре $j$, а $s_{i,j} = 0$ -- что задание $i$ не выполняется на процессоре $j$.
            

        \subsection*{Требуется:}

                Построить расписание $S$, при котором все задания $J$ будут выполнены на множестве процессоров $P$ без прерываний, с учетом ограниченных ресурсов, и не будет пересечений в использовании процессоров.

        \subsection*{Минимизируемый критерий:}
            \begin{itemize}
                \item В зависимости от остатка от деления на 2 контрольной суммы CRC32 от фамилии и инициалов выбирается один из следующих критериев:
                
                \begin{itemize}
                    \item \textbf{Критерий $K_1$ (разбалансированность расписания):}
                    
                    Разбалансированность расписания определяется как разность между максимальным и минимальным временем завершения заданий:
                    \begin{equation}
                        K_1 = T_{\text{max}} - T_{\text{min}},
                    \end{equation}
                    где $T_{\text{max}} = \max_{i \in \{1, \dots, m\}} c_i$ -- максимальное время завершения задания на любом из процессоров, а $T_{\text{min}} = \min_{i \in \{1, \dots, m\}} c_i$ -- минимальное время завершения задания на любом из процессоров.
                    
                    Необходимо минимизировать разбалансированность:
                    \begin{equation}
                        S^* = \arg \min_{S} K_1.
                    \end{equation}
                    
                    \item \textbf{Критерий $K_2$ (суммарное время ожидания):}
                    
                    Суммарное время ожидания определяется как сумма моментов завершения всех заданий:
                    \begin{equation}
                        K_2 = \sum_{i=1}^{n} c_i.
                    \end{equation}
                    
                    Необходимо минимизировать суммарное время ожидания:
                    \begin{equation}
                        S^* = \arg \min_{S} K_2.
                    \end{equation}
                \end{itemize}
            \end{itemize}

    \section*{Ограничения}

        \begin{itemize}
            \item Для каждого процессора $P_j \in P$ в любой момент времени он может выполнять не более одного задания.
            \item Каждое задание $J_i \in J$ должно быть выполнено только один раз и только на одном процессоре.
            \item Времена начала выполнения заданий $s_i$ должны быть выбраны так, чтобы не было конфликтов в использовании процессоров.
        \end{itemize}

\end{document}